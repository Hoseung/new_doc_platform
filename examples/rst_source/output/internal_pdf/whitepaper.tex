% LitePub Norm - LaTeX Template for PDF Generation
% Supports English + Korean (CJK) text with XeLaTeX
%
% Required: XeLaTeX engine with fontspec

\documentclass[11pt,a4paper]{article}

% Essential packages
\usepackage{fontspec}
\usepackage{xeCJK}
\usepackage{geometry}
\usepackage{hyperref}
\usepackage{graphicx}
\usepackage{longtable}
\usepackage{booktabs}
\usepackage{listings}
\usepackage{xcolor}
\usepackage{fancyhdr}
\usepackage{titlesec}
\usepackage{enumitem}
\usepackage{float}
\usepackage{array}
\usepackage{calc}

% Page geometry
\geometry{
  a4paper,
  margin=2.5cm,
  top=3cm,
  bottom=3cm,
  headheight=14pt
}

% Font configuration
% Main font (Latin)
\setmainfont{DejaVu Serif}[
  BoldFont=DejaVu Serif Bold,
  ItalicFont=DejaVu Serif Italic,
  BoldItalicFont=DejaVu Serif Bold Italic
]

% Sans font
\setsansfont{DejaVu Sans}[
  BoldFont=DejaVu Sans Bold,
  ItalicFont=DejaVu Sans Oblique
]

% Monospace font for code
\setmonofont{DejaVu Sans Mono}[
  Scale=0.9
]

% CJK configuration
% Preserve spaces in CJK text (important for Korean)
\xeCJKsetup{
  CJKspace=true,
  CJKecglue={\hskip 0.15em plus 0.05em minus 0.05em}
}

% CJK font (Korean)
% Try common Korean fonts, fallback to available CJK font
\setCJKmainfont{Noto Sans CJK KR}[
  BoldFont=Noto Sans CJK KR Bold
]
\setCJKsansfont{Noto Sans CJK KR}
\setCJKmonofont{Noto Sans Mono CJK KR}

% Hyperref configuration
\hypersetup{
  colorlinks=true,
  linkcolor=blue!60!black,
  urlcolor=blue!60!black,
  citecolor=green!50!black,
  pdfauthor={},
  pdftitle={},
  pdfsubject={},
  pdfkeywords={},
  bookmarksnumbered=true,
  bookmarksopen=true
}

% Headers and footers
\pagestyle{fancy}
\fancyhf{}
\fancyhead[L]{\leftmark}
\fancyhead[R]{\thepage}
\renewcommand{\headrulewidth}{0.4pt}

% Code listing style
\lstset{
  basicstyle=\ttfamily\small,
  breaklines=true,
  breakatwhitespace=true,
  numbers=left,
  numberstyle=\tiny\color{gray},
  frame=single,
  framesep=3pt,
  backgroundcolor=\color{gray!5},
  rulecolor=\color{gray!30},
  showstringspaces=false,
  tabsize=2,
  captionpos=b
}

% Section title formatting
\titleformat{\section}
  {\Large\bfseries}
  {\thesection}{1em}{}
\titleformat{\subsection}
  {\large\bfseries}
  {\thesubsection}{1em}{}
\titleformat{\subsubsection}
  {\normalsize\bfseries}
  {\thesubsubsection}{1em}{}

% Table styling
\renewcommand{\arraystretch}{1.3}

% Paragraph spacing
\setlength{\parindent}{0pt}
\setlength{\parskip}{0.5em}

% List customization
\setlist{noitemsep, topsep=0.5em}

% Custom commands
\newcommand{\placeholder}[1]{\textcolor{red}{\textbf{[#1]}}}

% Compatibility: Define \chapter for article class (used by some RST sources)
\makeatletter
\@ifundefined{chapter}{%
  % Handle both \chapter and \chapter* using unique names to avoid conflicts
  \newcommand{\@litepub@chapter}[1]{\section{#1}}
  \newcommand{\@litepub@schapter}[1]{\section*{#1}}
  \newcommand{\chapter}{\@ifstar\@litepub@schapter\@litepub@chapter}
  \providecommand{\chaptermark}[1]{\markboth{#1}{}}
}{}
\makeatother

% Pandoc-specific settings
\providecommand{\tightlist}{%
  \setlength{\itemsep}{0pt}\setlength{\parskip}{0pt}}

% Pandoc 3.x bounded image support
\providecommand{\pandocbounded}[1]{#1}

% Pandoc 3.x table width support (calc package \real for column widths)
\makeatletter
\providecommand{\real}[1]{\strip@pt\dimexpr #1pt\relax}
\makeatother

% Allow long tables to break across pages
\setlength{\LTpre}{0pt}
\setlength{\LTpost}{0pt}



% Document info

\begin{document}





\section{White Paper Title}\label{white-paper-title}

\textbf{An Example Technical Document}

\begin{description}
\tightlist
\item[Author]
Author Name
\item[Organization]
Organization
\item[Email]
\href{mailto:author@example.com}{\nolinkurl{author@example.com}}
\item[Date]
2025-11-23
\end{description}

\chapter*{Abstract}
\addcontentsline{toc}{chapter}{Abstract}

This is an example abstract for the white paper. It provides a brief
overview of the document\textquotesingle s contents, methodology, and
key findings. The abstract should be concise yet informative, typically
150-250 words.

This document demonstrates the use of Pandoc for professional technical
writing with automated figure and table generation using Python scripts.

한글도 완벽하게 지원됩니다! Korean text is fully supported!

\textbf{Keywords:} technical writing, pandoc, markdown, automation,
multilingual

\tableofcontents
\clearpage

chapters/01-introduction chapters/02-methodology chapters/03-results
chapters/04-conclusion chapters/05-korean-example
chapters/06-system-overview chapters/90-appendix-documentation

\subsection{Indices and tables}\label{indices-and-tables}

\begin{itemize}
\tightlist
\item
  \texttt{genindex}
\item
  \texttt{search}
\end{itemize}

\subsection{Introduction}\label{sec:introduction}

This document serves as an example white paper template using Pandoc and
Markdown. It demonstrates how to structure a professional technical
document with automated figure and table generation using Python
scripts.

\subsubsection{Background}\label{sec:background}

Technical writing often requires the integration of dynamically
generated content such as figures, tables, and data visualizations.
Traditional document preparation systems can make this process
cumbersome and error-prone. This template provides a modern,
reproducible approach to technical writing that combines:

\begin{itemize}
\tightlist
\item
  \textbf{Markdown} for easy-to-write, human-readable content
\item
  \textbf{Pandoc} for professional document conversion
\item
  \textbf{Python} for automated content generation
\item
  \textbf{Version control} for collaborative editing
\end{itemize}

As demonstrated in previous work {[}@example2024{]}, automated
documentation workflows significantly improve reproducibility and reduce
errors in scientific and technical publications.

\subsubsection{Objectives}\label{sec:objectives}

The primary objectives of this white paper template are:

\begin{enumerate}
\def\labelenumi{\arabic{enumi}.}
\tightlist
\item
  \textbf{Simplicity}: Enable writers to focus on content rather than
  formatting
\item
  \textbf{Reproducibility}: Ensure all figures and tables can be
  regenerated consistently
\item
  \textbf{Flexibility}: Support multiple output formats (PDF, HTML,
  DOCX) from the same source
\item
  \textbf{Professional Quality}: Produce publication-ready documents
  with proper numbering and cross-references
\end{enumerate}

\subsubsection{Document Structure}\label{sec:structure}

This white paper is organized as follows:

\begin{itemize}
\tightlist
\item
  \textbf{Chapter 1 (Introduction)}: Provides context and objectives
\item
  \textbf{Chapter 2 (Methodology)}: Describes the technical approach
\item
  \textbf{Chapter 3 (Results)}: Presents findings with figures and
  tables
\item
  \textbf{Chapter 4 (Conclusion)}: Summarizes key takeaways
\end{itemize}

Each chapter is maintained as a separate Markdown file in the
\texttt{chapters/} directory, making it easy to edit and manage content
modularly.

\subsubsection{Cross-References}\label{cross-references}

This template supports automatic cross-referencing of:

\begin{itemize}
\tightlist
\item
  \textbf{Sections}: Reference Section @sec:methodology for technical
  details
\item
  \textbf{Figures}: See Figure @fig:example\_plot for a visual example
\item
  \textbf{Tables}: Refer to Table @tbl:comparison for data comparison
\item
  \textbf{Equations}: Equation @eq:example demonstrates mathematical
  notation
\end{itemize}

These cross-references are automatically updated when the document is
built, ensuring consistency throughout the paper.

\subsection{Methodology}\label{sec:methodology}

This chapter describes the technical approach and methodology used in
this white paper template system.

\subsubsection{System Architecture}\label{sec:architecture}

The documentation system consists of several integrated components:

\begin{enumerate}
\def\labelenumi{\arabic{enumi}.}
\tightlist
\item
  \textbf{Content Layer}: Markdown files containing the written content
\item
  \textbf{Generation Layer}: Python scripts that create figures and
  tables
\item
  \textbf{Build Layer}: Pandoc and Make for document compilation
\item
  \textbf{Output Layer}: Generated documents in various formats
\end{enumerate}

This modular architecture ensures that each component can be developed
and maintained independently, as discussed by @examplebook2023.

\subsubsection{Content Generation}\label{sec:content-generation}

\paragraph{Figure Generation}\label{sec:figures}

Figures are generated programmatically using Python's matplotlib
library. The generation process follows these steps:

\begin{enumerate}
\def\labelenumi{\arabic{enumi}.}
\tightlist
\item
  Define the data and visualization parameters
\item
  Create the plot using matplotlib
\item
  Apply styling for professional appearance
\item
  Save to the \texttt{../\_static/figures/} directory at high resolution
  (300 DPI)
\end{enumerate}

For example, the sine wave shown in Figure @fig:example\_plot is
generated using:

\begin{verbatim}
x = np.linspace(0, 10, 1000)
y = np.sin(x)
plt.plot(x, y)
plt.savefig('../_static/figures/example_line_plot.png', dpi=300)
\end{verbatim}

\protect\phantomsection\label{fig:example_plot}
\begin{figure}
\centering
\pandocbounded{\includegraphics[keepaspectratio]{/home/hoseung/Work/DMS_eval/new_doc_platform/examples/rst_source/config/../_static/figures/example_line_plot.png}}
\caption{}\label{fig:example_plot}
\end{figure}

\paragraph{Table Generation}\label{sec:tables}

Tables are generated using pandas DataFrames and exported as Markdown
format. This ensures:

\begin{itemize}
\tightlist
\item
  \textbf{Data integrity}: Tables reflect actual data values
\item
  \textbf{Consistency}: Formatting is uniform across all tables
\item
  \textbf{Reproducibility}: Tables can be regenerated with updated data
\end{itemize}

Table @tbl:comparison shows an example of a comparison table generated
from the \texttt{generate\_tables.py} script.

\begin{longtable}[]{@{}lllll@{}}
\caption{Comparison of different methodological
approaches}\tabularnewline
\toprule\noalign{}
Method & Accuracy (\%) & Speed (ms) & Memory (MB) & Complexity \\
\midrule\noalign{}
\endfirsthead
\toprule\noalign{}
Method & Accuracy (\%) & Speed (ms) & Memory (MB) & Complexity \\
\midrule\noalign{}
\endhead
\bottomrule\noalign{}
\endlastfoot
Method A & 92.5 & 12.3 & 256 & Low \\
Method B & 94.3 & 15.7 & 512 & Medium \\
Method C & 91.8 & 10.2 & 128 & Low \\
Method D & 95.1 & 18.4 & 1024 & High \\
\end{longtable}

\subsubsection{Mathematical Notation}\label{sec:mathematics}

The system supports mathematical equations using LaTeX syntax. For
instance, the Gaussian function is defined as:

\[f(x) = \frac{1}{\sigma\sqrt{2\pi}} e^{-\frac{1}{2}\left(\frac{x-\mu}{\sigma}\right)^2}\]

\{\#eq:gaussian\}

where \(\mu\) is the mean and \(\sigma\) is the standard deviation.

More complex equations can also be included. The general form of a
neural network layer computation is:

\[\mathbf{y} = \sigma(W\mathbf{x} + \mathbf{b})\]

\{\#eq:neural\}

where \(W\) is the weight matrix, \(\mathbf{b}\) is the bias vector, and
\(\sigma\) is the activation function.

\subsubsection{Build Process}\label{sec:build}

The document build process is automated using a Makefile:

\begin{verbatim}
make all      # Generate figures, tables, and PDF
make pdf      # Build PDF only
make html     # Build HTML version
make clean    # Remove generated files
\end{verbatim}

This automation ensures reproducibility and simplifies the workflow for
authors, as noted in recent conference proceedings
{[}@exampleconf2024{]}.

\subsection{Results}\label{sec:results}

This chapter presents the results obtained using the methodologies
described in Section @sec:methodology. All figures and tables are
generated automatically from Python scripts to ensure reproducibility.

\textbf{Automatic Table Generation}

Tables in this chapter are generated using the TableManager system. They
are automatically created by running:

\begin{verbatim}
python scripts/generate_tables_v2.py
\end{verbatim}

Tables are saved to \texttt{../\_static/tables/chapter-03/} and can be
easily included or updated. See
\href{../TABLE_WORKFLOW.md}{TABLE\_WORKFLOW.md} for details.

\subsubsection{Visualization Results}\label{sec:visualization}

\paragraph{Line Plots}\label{sec:line-plots}

Figure @fig:scatter shows the relationship between variables with a
fitted trend line. The scatter plot reveals a strong linear correlation
with coefficient of determination R² = 0.89.

\protect\phantomsection\label{fig:scatter}
\begin{figure}
\centering
\pandocbounded{\includegraphics[keepaspectratio]{/home/hoseung/Work/DMS_eval/new_doc_platform/examples/rst_source/config/../_static/figures/example_scatter_plot.png}}
\caption{}\label{fig:scatter}
\end{figure}

\paragraph{Categorical Comparisons}\label{sec:bar-charts}

Figure @fig:bars presents a categorical comparison across five different
categories. Category D shows the highest value at 78 units, while
Category E demonstrates the lowest at 32 units.

\protect\phantomsection\label{fig:bars}
\begin{figure}
\centering
\pandocbounded{\includegraphics[keepaspectratio]{/home/hoseung/Work/DMS_eval/new_doc_platform/examples/rst_source/config/../_static/figures/example_bar_chart.png}}
\caption{}\label{fig:bars}
\end{figure}

\paragraph{Distribution Analysis}\label{sec:distributions}

The histogram in Figure @fig:histogram illustrates the distribution of
measured values. The data closely follows a normal distribution with
mean μ = 100.2 and standard deviation σ = 14.8.

\protect\phantomsection\label{fig:histogram}
\begin{figure}
\centering
\pandocbounded{\includegraphics[keepaspectratio]{/home/hoseung/Work/DMS_eval/new_doc_platform/examples/rst_source/config/../_static/figures/example_histogram.png}}
\caption{}\label{fig:histogram}
\end{figure}

\paragraph{Correlation Analysis}\label{sec:correlation}

Figure @fig:heatmap presents the correlation matrix between six
different variables. Strong positive correlations (\textgreater0.7) are
observed between Variables 1 and 3, while Variables 2 and 5 show weak
correlation (\textless0.3).

\protect\phantomsection\label{fig:heatmap}
\begin{figure}
\centering
\pandocbounded{\includegraphics[keepaspectratio]{/home/hoseung/Work/DMS_eval/new_doc_platform/examples/rst_source/config/../_static/figures/example_heatmap.png}}
\caption{}\label{fig:heatmap}
\end{figure}

\subsubsection{Tabular Results}\label{sec:tabular}

\paragraph{Experimental Data}\label{sec:experimental-data}

Table @tbl:experimental presents the experimental results across five
different conditions. The data shows a clear trend of increasing yield
with temperature, from 78.3\% at 20.5°C to 92.5\% at 40.5°C.

\begin{longtable}[]{@{}
  >{\raggedright\arraybackslash}p{(\linewidth - 8\tabcolsep) * \real{0.1688}}
  >{\raggedright\arraybackslash}p{(\linewidth - 8\tabcolsep) * \real{0.2468}}
  >{\raggedright\arraybackslash}p{(\linewidth - 8\tabcolsep) * \real{0.2208}}
  >{\raggedright\arraybackslash}p{(\linewidth - 8\tabcolsep) * \real{0.1558}}
  >{\raggedright\arraybackslash}p{(\linewidth - 8\tabcolsep) * \real{0.1688}}@{}}
\caption{Experimental results under different temperature and pressure
conditions}\tabularnewline
\toprule\noalign{}
\begin{minipage}[b]{\linewidth}\raggedright
Experiment
\end{minipage} & \begin{minipage}[b]{\linewidth}\raggedright
Temperature (°C)
\end{minipage} & \begin{minipage}[b]{\linewidth}\raggedright
Pressure (kPa)
\end{minipage} & \begin{minipage}[b]{\linewidth}\raggedright
Yield (\%)
\end{minipage} & \begin{minipage}[b]{\linewidth}\raggedright
Time (min)
\end{minipage} \\
\midrule\noalign{}
\endfirsthead
\toprule\noalign{}
\begin{minipage}[b]{\linewidth}\raggedright
Experiment
\end{minipage} & \begin{minipage}[b]{\linewidth}\raggedright
Temperature (°C)
\end{minipage} & \begin{minipage}[b]{\linewidth}\raggedright
Pressure (kPa)
\end{minipage} & \begin{minipage}[b]{\linewidth}\raggedright
Yield (\%)
\end{minipage} & \begin{minipage}[b]{\linewidth}\raggedright
Time (min)
\end{minipage} \\
\midrule\noalign{}
\endhead
\bottomrule\noalign{}
\endlastfoot
Exp-1 & 20.5 & 101.3 & 78.3 & 45 \\
Exp-2 & 25.0 & 105.2 & 82.1 & 42 \\
Exp-3 & 30.5 & 110.1 & 85.7 & 38 \\
Exp-4 & 35.0 & 115.8 & 89.2 & 35 \\
Exp-5 & 40.5 & 121.3 & 92.5 & 32 \\
\end{longtable}

\paragraph{Statistical Summary}\label{sec:statistics}

Table @tbl:stats provides a statistical summary of three datasets.
Dataset B exhibits the highest mean value (119.64) and largest standard
deviation (20.31), indicating greater variability compared to the other
datasets.

\begin{longtable}[]{@{}llllll@{}}
\caption{Statistical summary of experimental datasets}\tabularnewline
\toprule\noalign{}
Dataset & Mean & Std Dev & Min & Max & Median \\
\midrule\noalign{}
\endfirsthead
\toprule\noalign{}
Dataset & Mean & Std Dev & Min & Max & Median \\
\midrule\noalign{}
\endhead
\bottomrule\noalign{}
\endlastfoot
Dataset A & 100.18 & 15.07 & 60.42 & 139.62 & 100.35 \\
Dataset B & 119.64 & 20.31 & 61.70 & 176.47 & 120.17 \\
Dataset C & 89.82 & 9.91 & 62.04 & 116.93 & 89.95 \\
\end{longtable}

\paragraph{Performance Metrics}\label{sec:performance}

Model performance across different data splits is summarized in Table
@tbl:performance. The results demonstrate good generalization, with test
set performance (F1-Score: 0.915) closely matching validation set
performance (F1-Score: 0.918).

\begin{longtable}[]{@{}llll@{}}
\caption{Performance metrics evaluated on training, validation, and test
sets}\tabularnewline
\toprule\noalign{}
Metric & Training Set & Validation Set & Test Set \\
\midrule\noalign{}
\endfirsthead
\toprule\noalign{}
Metric & Training Set & Validation Set & Test Set \\
\midrule\noalign{}
\endhead
\bottomrule\noalign{}
\endlastfoot
Precision & 0.953 & 0.921 & 0.918 \\
Recall & 0.947 & 0.915 & 0.912 \\
F1-Score & 0.950 & 0.918 & 0.915 \\
Accuracy & 0.948 & 0.916 & 0.913 \\
AUC-ROC & 0.982 & 0.965 & 0.961 \\
\end{longtable}

\subsubsection{Key Findings}\label{sec:findings}

The analysis presented in this chapter yields several important
findings:

\begin{enumerate}
\def\labelenumi{\arabic{enumi}.}
\tightlist
\item
  \textbf{Strong Linear Relationships}: As shown in Figure @fig:scatter,
  the linear model provides excellent fit to the experimental data
\item
  \textbf{Temperature Dependence}: Table @tbl:experimental demonstrates
  a clear positive correlation between temperature and yield
\item
  \textbf{Model Robustness}: Performance metrics in Table
  @tbl:performance indicate minimal overfitting
\item
  \textbf{Variable Independence}: The correlation analysis (Figure
  @fig:heatmap) reveals which variables can be treated independently
\end{enumerate}

These findings support the methodology outlined in Chapter
@sec:methodology and provide a foundation for the conclusions drawn in
Chapter @sec:conclusion.

\subsection{Conclusion}\label{sec:conclusion}

This white paper has demonstrated a comprehensive approach to technical
document authoring using Markdown, Pandoc, and Python for automated
content generation.

\subsubsection{Summary of Contributions}\label{sec:summary}

The key contributions of this template system include:

\begin{enumerate}
\def\labelenumi{\arabic{enumi}.}
\tightlist
\item
  \textbf{Automated Workflow}: Integration of Python-based figure and
  table generation with Pandoc document compilation
\item
  \textbf{Reproducibility}: All content can be regenerated consistently
  from source data
\item
  \textbf{Professional Output}: High-quality documents suitable for
  academic and technical publications
\item
  \textbf{Multi-format Support}: Single source generates PDF, HTML, and
  DOCX outputs
\item
  \textbf{Version Control Friendly}: Plain text format enables effective
  collaboration using Git
\end{enumerate}

As discussed in Section @sec:methodology, the modular architecture
separates content creation, visualization, and document compilation into
distinct, manageable components.

\subsubsection{Results Overview}\label{sec:results-overview}

The results presented in Chapter @sec:results demonstrate the
capabilities of this system:

\begin{itemize}
\tightlist
\item
  \textbf{Visualizations}: High-quality figures (Figures
  @fig:example\_plot, @fig:scatter, @fig:bars, @fig:histogram,
  @fig:heatmap) generated at publication quality (300 DPI)
\item
  \textbf{Tables}: Automatically formatted tables (Tables
  @tbl:comparison, @tbl:experimental, @tbl:stats, @tbl:performance) with
  consistent styling
\item
  \textbf{Cross-references}: Seamless linking between sections, figures,
  tables, and equations
\item
  \textbf{Mathematical Notation}: Professional typesetting of equations
  (e.g., Equations @eq:gaussian, @eq:neural)
\end{itemize}

\subsubsection{Best Practices}\label{sec:best-practices}

Based on the implementation of this system, several best practices
emerge:

\paragraph{Content Management}\label{content-management}

\begin{itemize}
\tightlist
\item
  \textbf{Modular Chapters}: Maintain each chapter in a separate file
  for easier editing and collaboration
\item
  \textbf{Meaningful IDs}: Use descriptive identifiers for
  cross-references (e.g., \texttt{\#fig:methodology-flowchart} rather
  than \texttt{\#fig1})
\item
  \textbf{Version Control}: Commit source files (.md, .py, .yaml) but
  exclude generated outputs (.pdf, .png)
\end{itemize}

\paragraph{Figure Generation}\label{figure-generation-1}

\begin{itemize}
\tightlist
\item
  \textbf{High Resolution}: Save figures at 300 DPI or higher for
  publication quality
\item
  \textbf{Consistent Styling}: Define standard plot styles in a
  configuration file or module
\item
  \textbf{Descriptive Names}: Use clear, descriptive filenames for
  generated figures
\end{itemize}

\paragraph{Table Generation}\label{table-generation-1}

\begin{itemize}
\tightlist
\item
  \textbf{Data-Driven}: Generate tables from actual data rather than
  manual entry
\item
  \textbf{Standard Format}: Use pandas DataFrames for consistency across
  all tables
\item
  \textbf{Precision Control}: Specify appropriate decimal places for
  numerical values
\end{itemize}

\paragraph{Build Automation}\label{build-automation}

\begin{itemize}
\tightlist
\item
  \textbf{Makefile Targets}: Define clear targets for common build
  operations
\item
  \textbf{Dependency Tracking}: Ensure figures and tables are
  regenerated before document compilation
\item
  \textbf{Clean Operations}: Provide targets to remove generated files
  and start fresh
\end{itemize}

\subsubsection{Limitations and Future Work}\label{sec:limitations}

While this system provides a robust foundation for technical writing,
several areas merit future development:

\begin{enumerate}
\def\labelenumi{\arabic{enumi}.}
\tightlist
\item
  \textbf{Template Customization}: Additional LaTeX templates for
  specific publication styles
\item
  \textbf{Interactive HTML}: Enhanced HTML output with interactive plots
  using Plotly or Bokeh
\item
  \textbf{Collaborative Editing}: Integration with real-time
  collaboration tools
\item
  \textbf{Automated Testing}: Unit tests for Python scripts to ensure
  correct figure/table generation
\item
  \textbf{Bibliography Management}: Tighter integration with reference
  management systems (Zotero, Mendeley)
\end{enumerate}

\subsubsection{Final Remarks}\label{sec:final-remarks}

This white paper template provides a modern, efficient approach to
technical writing that addresses the common challenges of
reproducibility, version control, and multi-format output. By combining
the simplicity of Markdown with the power of Pandoc and Python, authors
can focus on content creation while maintaining professional standards.

The complete source code and documentation for this template are
available in the project repository. Users are encouraged to customize
and extend the system to meet their specific requirements.

As noted in the Pandoc documentation {[}@exampleweb2024{]}, the future
of technical writing lies in formats that are both human-readable and
machine-processable. This template embraces that philosophy while
maintaining compatibility with traditional publishing workflows.

\subsubsection{Acknowledgments}\label{sec:acknowledgments}

This template builds upon the excellent work of the Pandoc development
team, the Python scientific computing community, and countless
contributors to open-source documentation tools. Special thanks to the
maintainers of pandoc-crossref for enabling sophisticated
cross-referencing capabilities.

\begin{center}\rule{0.5\linewidth}{0.5pt}\end{center}

\textbf{For questions or contributions, please refer to the project
repository documentation.}

\subsection{한글 사용 예제 / Korean Example}\label{sec:korean}

이 장에서는 한글과 영어를 함께 사용하는 방법을 보여줍니다.

This chapter demonstrates how to use Korean and English together in your
white paper.

\subsubsection{혼합 언어 작성 / Mixed Language Writing}\label{sec:mixed}

\paragraph{기본 텍스트 / Basic
Text}\label{uxae30uxbcf8-uxd14duxc2a4uxd2b8-basic-text}

한글과 영어를 자연스럽게 섞어서 사용할 수 있습니다. You can naturally
mix Korean and English text. 예를 들어, \textbf{머신러닝} (machine
learning)이나 \textbf{딥러닝} (deep learning)과 같은 기술 용어를 함께
사용할 수 있습니다.

기술 문서에서는 종종 한글 설명과 영어 원어를 병기합니다:

\begin{itemize}
\tightlist
\item
  \textbf{인공지능} (Artificial Intelligence, AI)
\item
  \textbf{자연어 처리} (Natural Language Processing, NLP)
\item
  \textbf{컴퓨터 비전} (Computer Vision, CV)
\end{itemize}

\paragraph{수식과 한글 / Equations with
Korean}\label{uxc218uxc2dduxacfc-uxd55cuxae00-equations-with-korean}

한글 설명과 함께 수학 공식을 사용할 수 있습니다. Mathematical equations
can be used with Korean explanations.

정규분포의 확률밀도함수는 다음과 같이 정의됩니다:

\[f(x) = \frac{1}{\sigma\sqrt{2\pi}} e^{-\frac{1}{2}\left(\frac{x-\mu}{\sigma}\right)^2}\]

\{\#eq:korean-normal\}

여기서 \(\mu\)는 평균 (mean), \(\sigma\)는 표준편차 (standard
deviation)입니다.

\subsubsection{표와 그림 / Tables and Figures}\label{sec:korean-tables}

\paragraph{한글 표 / Korean
Tables}\label{uxd55cuxae00-uxd45c-korean-tables}

한글로 작성된 표의 예시입니다. Here is an example of a table with Korean
text:

\begin{longtable}[]{@{}
  >{\raggedright\arraybackslash}p{(\linewidth - 12\tabcolsep) * \real{0.1364}}
  >{\raggedright\arraybackslash}p{(\linewidth - 12\tabcolsep) * \real{0.1364}}
  >{\raggedright\arraybackslash}p{(\linewidth - 12\tabcolsep) * \real{0.1477}}
  >{\raggedright\arraybackslash}p{(\linewidth - 12\tabcolsep) * \real{0.1250}}
  >{\raggedright\arraybackslash}p{(\linewidth - 12\tabcolsep) * \real{0.1477}}
  >{\raggedright\arraybackslash}p{(\linewidth - 12\tabcolsep) * \real{0.1591}}
  >{\raggedright\arraybackslash}p{(\linewidth - 12\tabcolsep) * \real{0.0909}}@{}}
\caption{온도에 따른 실험 결과 / Experimental results by
temperature}\tabularnewline
\toprule\noalign{}
\begin{minipage}[b]{\linewidth}\raggedright
실험 번호
\end{minipage} & \begin{minipage}[b]{\linewidth}\raggedright
온도 (°C)
\end{minipage} & \begin{minipage}[b]{\linewidth}\raggedright
압력 (kPa)
\end{minipage} & \begin{minipage}[b]{\linewidth}\raggedright
수율 (\%)
\end{minipage} & \begin{minipage}[b]{\linewidth}\raggedright
Experiment
\end{minipage} & \begin{minipage}[b]{\linewidth}\raggedright
Temperature
\end{minipage} & \begin{minipage}[b]{\linewidth}\raggedright
Yield
\end{minipage} \\
\midrule\noalign{}
\endfirsthead
\toprule\noalign{}
\begin{minipage}[b]{\linewidth}\raggedright
실험 번호
\end{minipage} & \begin{minipage}[b]{\linewidth}\raggedright
온도 (°C)
\end{minipage} & \begin{minipage}[b]{\linewidth}\raggedright
압력 (kPa)
\end{minipage} & \begin{minipage}[b]{\linewidth}\raggedright
수율 (\%)
\end{minipage} & \begin{minipage}[b]{\linewidth}\raggedright
Experiment
\end{minipage} & \begin{minipage}[b]{\linewidth}\raggedright
Temperature
\end{minipage} & \begin{minipage}[b]{\linewidth}\raggedright
Yield
\end{minipage} \\
\midrule\noalign{}
\endhead
\bottomrule\noalign{}
\endlastfoot
실험-1 & 20.5 & 101.3 & 78.3 & Exp-1 & 20.5°C & 78.3\% \\
실험-2 & 25.0 & 105.2 & 82.1 & Exp-2 & 25.0°C & 82.1\% \\
실험-3 & 30.5 & 110.1 & 85.7 & Exp-3 & 30.5°C & 85.7\% \\
실험-4 & 35.0 & 115.8 & 89.2 & Exp-4 & 35.0°C & 89.2\% \\
실험-5 & 40.5 & 121.3 & 92.5 & Exp-5 & 40.5°C & 92.5\% \\
\end{longtable}

표 @tbl:korean-exp 에서 볼 수 있듯이, 온도가 증가함에 따라 수율도 함께
증가했습니다.

As shown in Table @tbl:korean-exp, the yield increased with temperature.

\paragraph{방법론 비교 표 / Method Comparison
Table}\label{uxbc29uxbc95uxb860-uxbe44uxad50-uxd45c-method-comparison-table}

\begin{longtable}[]{@{}lllll@{}}
\caption{다양한 방법론의 성능 비교 / Performance comparison of different
methods}\tabularnewline
\toprule\noalign{}
방법 & 정확도 (\%) & 속도 (ms) & 메모리 (MB) & 복잡도 \\
\midrule\noalign{}
\endfirsthead
\toprule\noalign{}
방법 & 정확도 (\%) & 속도 (ms) & 메모리 (MB) & 복잡도 \\
\midrule\noalign{}
\endhead
\bottomrule\noalign{}
\endlastfoot
방법 A & 92.5 & 12.3 & 256 & 낮음 \\
방법 B & 94.3 & 15.7 & 512 & 중간 \\
방법 C & 91.8 & 10.2 & 128 & 낮음 \\
방법 D & 95.1 & 18.4 & 1024 & 높음 \\
\end{longtable}

\subsubsection{교차 참조 / Cross-References}\label{sec:korean-refs}

\paragraph{섹션 참조 / Section
References}\label{uxc139uxc158-uxcc38uxc870-section-references}

한글로 된 섹션도 쉽게 참조할 수 있습니다. Korean sections can be easily
referenced.

\begin{itemize}
\tightlist
\item
  서론은 Section @sec:introduction 을 참조하세요
\item
  방법론은 Section @sec:methodology 를 참조하세요
\item
  이 장의 표 섹션은 Section @sec:korean-tables 에 있습니다
\end{itemize}

\paragraph{그림 참조 / Figure
References}\label{uxadf8uxb9bc-uxcc38uxc870-figure-references}

영어 장의 그림도 참조할 수 있습니다:

\begin{itemize}
\tightlist
\item
  Figure @fig:example\_plot 는 삼각함수 그래프를 보여줍니다
\item
  Figure @fig:scatter 는 산점도와 회귀선을 보여줍니다
\item
  Figure @fig:heatmap 은 상관관계 히트맵입니다
\end{itemize}

\paragraph{수식 참조 / Equation
References}\label{uxc218uxc2dd-uxcc38uxc870-equation-references}

Equation @eq:korean-normal 은 정규분포를 나타내며, Equation @eq:gaussian
은 같은 공식의 영어 버전입니다.

\subsubsection{인용 / Citations}\label{sec:korean-citations}

한글로 작성할 때도 인용을 동일하게 사용할 수 있습니다. Citations work
the same way in Korean text.

최근 연구에 따르면 {[}@example2024{]}, 자동화된 문서 작성 시스템이
재현성을 크게 향상시킨다고 합니다.

여러 문헌을 동시에 인용할 수도 있습니다 {[}@example2024;
@examplebook2023{]}.

\subsubsection{코드 블록 / Code Blocks}\label{sec:korean-code}

한글 주석이 포함된 코드 예제:

\begin{verbatim}
# 한글 주석도 잘 표시됩니다
import numpy as np
import matplotlib.pyplot as plt

# 데이터 생성
x = np.linspace(0, 10, 100)
y = np.sin(x)

# 그래프 그리기
plt.plot(x, y)
plt.xlabel('시간 (초)')  # X축 레이블
plt.ylabel('진폭')       # Y축 레이블
plt.title('사인파 그래프')
plt.show()
\end{verbatim}

\subsubsection{목록 / Lists}\label{sec:korean-lists}

\paragraph{순서 있는 목록 / Ordered
List}\label{uxc21cuxc11c-uxc788uxb294-uxbaa9uxb85d-ordered-list}

연구 진행 단계:

\begin{enumerate}
\def\labelenumi{\arabic{enumi}.}
\tightlist
\item
  \textbf{문헌 조사} (Literature Review)

  \begin{itemize}
  \tightlist
  \item
    관련 연구 검색
  \item
    선행 연구 분석
  \end{itemize}
\item
  \textbf{실험 설계} (Experimental Design)

  \begin{itemize}
  \tightlist
  \item
    변수 선정
  \item
    측정 방법 결정
  \end{itemize}
\item
  \textbf{데이터 수집} (Data Collection)

  \begin{itemize}
  \tightlist
  \item
    실험 수행
  \item
    결과 기록
  \end{itemize}
\item
  \textbf{분석 및 해석} (Analysis and Interpretation)

  \begin{itemize}
  \tightlist
  \item
    통계 분석
  \item
    결과 해석
  \end{itemize}
\end{enumerate}

\paragraph{순서 없는 목록 / Unordered
List}\label{uxc21cuxc11c-uxc5c6uxb294-uxbaa9uxb85d-unordered-list}

주요 기여사항:

\begin{itemize}
\tightlist
\item
  \textbf{재현성 향상}: 모든 그림과 표를 자동으로 생성
\item
  \textbf{다국어 지원}: 한글, 영어, 일본어, 중국어 등 지원
\item
  \textbf{버전 관리}: Git을 통한 효율적인 협업
\item
  \textbf{다양한 출력 형식}: PDF, HTML, DOCX 등
\end{itemize}

\subsubsection{강조와 서식 / Emphasis and
Formatting}\label{sec:korean-formatting}

한글에서도 다양한 서식을 사용할 수 있습니다:

\begin{itemize}
\tightlist
\item
  \textbf{굵게} (bold)
\item
  \emph{기울임} (italic)
\item
  {[}STRIKEOUT:취소선{]} (strikethrough)
\item
  \texttt{코드} (inline code)
\item
  \href{https://example.com}{링크} (link)
\end{itemize}

\subsubsection{주요 결과 / Key Findings}\label{sec:korean-findings}

이 예제 장에서 다룬 내용:

\begin{enumerate}
\def\labelenumi{\arabic{enumi}.}
\tightlist
\item
  \textbf{혼합 언어 작성}: 한글과 영어를 자연스럽게 섞어 사용
\item
  \textbf{표와 그림}: 한글 레이블과 캡션 사용 (Tables @tbl:korean-exp,
  @tbl:korean-methods)
\item
  \textbf{수식}: 한글 설명과 함께 수학 공식 사용 (Equation
  @eq:korean-normal)
\item
  \textbf{교차 참조}: 한글 텍스트에서 섹션, 그림, 표 참조
\item
  \textbf{코드}: 한글 주석이 포함된 코드 블록
\item
  \textbf{서식}: 다양한 마크다운 서식 기능
\end{enumerate}

\subsubsection{결론 / Conclusion}\label{sec:korean-conclusion}

Pandoc과 XeLaTeX를 사용하면 한글과 영어를 완벽하게 혼용할 수 있습니다.
Using Pandoc with XeLaTeX enables perfect mixing of Korean and English.

한글 폰트가 제대로 설치되어 있다면, 별도의 추가 설정 없이도 한글 문서를
작성할 수 있습니다. With proper Korean fonts installed, you can write
Korean documents without additional configuration.

더 자세한 정보는 \href{../KOREAN_SUPPORT.md}{KOREAN\_SUPPORT.md} 문서를
참조하세요. For more information, see the
\href{../KOREAN_SUPPORT.md}{KOREAN\_SUPPORT.md} guide.

\subsection{System Overview and Use Cases}\label{sec:system-overview}

This chapter describes the design philosophy, features, and practical
use cases of this documentation system.

\subsubsection{Design Philosophy}\label{sec:design-philosophy}

This documentation system is built around four core principles that
address common pain points in technical writing:

\begin{description}
\item[\textbf{Easy to Edit}]
Content is written in plain text formats (Markdown or reStructuredText)
that require no special software. Any text editor works, and the syntax
is human-readable even without rendering.
\item[\textbf{Easy to Automate}]
Python scripts generate figures and tables programmatically. When
underlying data changes, a single command regenerates all derived
content. This ensures reproducibility and eliminates manual copy-paste
errors.
\item[\textbf{Easy to Render}]
A single source produces multiple output formats: PDF for printing, HTML
for web viewing, DOCX for collaboration with Word users, and EPUB for
e-readers. The Makefile handles all conversion automatically.
\item[\textbf{Easy to Process}]
Plain text sources with semantic markup are ideal for processing by
language models and other automated tools. The structured organization
and metadata-rich tables preserve context for AI-assisted workflows.
\end{description}

\subsubsection{Key Features}\label{sec:key-features}

\subparagraph{Dual Build System}\label{dual-build-system}

The system supports two documentation toolchains:

\begin{enumerate}
\def\labelenumi{\arabic{enumi}.}
\tightlist
\item
  \textbf{Pandoc workflow}: Write in Markdown, convert to PDF/HTML/DOCX
  using Pandoc with pandoc-crossref for cross-references.
\item
  \textbf{Sphinx workflow}: Write in reStructuredText, build with Sphinx
  for full-featured HTML documentation with search, PDF via LaTeX, and
  EPUB.
\end{enumerate}

Both workflows share the same Python scripts for figure and table
generation, allowing you to choose the toolchain that best fits your
needs.

\begin{verbatim}
# Pandoc workflow
make all          # Build PDF and HTML
make pdf          # Build PDF only

# Sphinx workflow
make sphinx-html  # Build HTML documentation
make sphinx-pdf   # Build PDF via LaTeX
\end{verbatim}

\subparagraph{Automated Content
Generation}\label{automated-content-generation}

Figures and tables are generated by Python scripts rather than created
manually:

\textbf{Figures}: The \texttt{generate\_figures.py} script creates
publication-quality plots using matplotlib at 300 DPI. When data or
visualization requirements change, regenerating figures is a single
command.

\textbf{Tables}: The \texttt{TableManager} class creates tables with
YAML frontmatter containing metadata (caption, label, description). The
preprocessor automatically replaces placeholders like
\texttt{\{\{table:tbl:03-results\}\}} with actual table content during
build.

\begin{verbatim}
from scripts.table_manager import TableManager

tm = TableManager(chapter=3)
tm.save_table(
    dataframe,
    name="experimental-results",
    caption="Experimental Results Summary",
    description="Results from the primary experiment series."
)
\end{verbatim}

\subparagraph{Self-Contained Tables}\label{self-contained-tables}

Each generated table is a self-contained file with YAML frontmatter:

\begin{verbatim}
---
label: tbl:03-experimental-results
caption: Experimental Results Summary
chapter: 3
description: Results from the primary experiment series.
generated: 2025-01-15T10:30:00
---

| Metric | Value | Unit |
|--------|-------|------|
| Accuracy | 95.2 | % |
| Precision | 94.8 | % |
\end{verbatim}

This metadata travels with the table, providing context even when
processed by external tools or language models.

\subparagraph{Korean/CJK Support}\label{koreancjk-support}

Full support for Korean, Japanese, and Chinese text:

\begin{itemize}
\tightlist
\item
  Pre-configured Noto CJK fonts for PDF output
\item
  Korean matplotlib utilities for properly rendered plot labels
\item
  Mixed-language documents work seamlessly
\item
  UTF-8 throughout the entire pipeline
\end{itemize}

\begin{verbatim}
from scripts.utils.korean_plot import setup_korean_font

setup_korean_font()
plt.xlabel('측정값 (Measurements)')
plt.title('실험 결과 분석')
\end{verbatim}

\subparagraph{Adaptive Toolchain}\label{adaptive-toolchain}

The Makefile automatically detects your Pandoc version and adjusts
accordingly:

\begin{itemize}
\tightlist
\item
  Uses \texttt{-\/-citeproc} for Pandoc 2.11+ or
  \texttt{-\/-filter\ pandoc-citeproc} for older versions
\item
  Gracefully degrades when optional filters (pandoc-crossref) are
  unavailable
\item
  Provides warnings but continues building when possible
\end{itemize}

This ensures the system works across different environments without
manual configuration.

\subsubsection{Use Cases}\label{sec:use-cases}

\subparagraph{Technical Reports and White
Papers}\label{technical-reports-and-white-papers}

The primary use case: writing technical documents where figures and
tables derive from data analysis.

\textbf{Workflow}:

\begin{enumerate}
\def\labelenumi{\arabic{enumi}.}
\tightlist
\item
  Perform analysis in Python, generating figures and tables
\item
  Write narrative content in Markdown/RST
\item
  Reference figures and tables using cross-reference syntax
\item
  Build the complete document with \texttt{make\ pdf}
\end{enumerate}

When the analysis updates, regenerate content and rebuild. The document
stays synchronized with your data.

\subparagraph{Reproducible Research}\label{reproducible-research}

For academic papers or research reports where reproducibility matters:

\begin{itemize}
\tightlist
\item
  All figures regenerable from source data via Python scripts
\item
  Git tracks both content and generation scripts
\item
  Anyone can clone the repository and rebuild the exact same document
\item
  Changes to methodology are reflected automatically in output
\end{itemize}

\subparagraph{LLM-Assisted
Documentation}\label{llm-assisted-documentation}

The plain text format and structured organization make this system ideal
for AI-assisted workflows:

\textbf{Content generation}: Language models can read the source files,
understand the document structure, and generate new sections that follow
existing patterns.

\textbf{Data analysis}: LLMs can write Python scripts that generate
figures and tables, which integrate seamlessly into the document.

\textbf{Review and editing}: The semantic markup (cross-references,
labeled sections) provides context that helps LLMs understand
relationships between document parts.

\textbf{Automation}: LLM agents can trigger build commands, check for
errors, and iterate on content.

Example LLM workflow:

\begin{enumerate}
\def\labelenumi{\arabic{enumi}.}
\tightlist
\item
  Provide the LLM with source files for context
\item
  Request new analysis or content
\item
  LLM generates Python scripts and/or document sections
\item
  Build and review the output
\item
  Iterate as needed
\end{enumerate}

\subparagraph{Multi-Format Publishing}\label{multi-format-publishing}

When the same content must reach different audiences:

\begin{itemize}
\tightlist
\item
  \textbf{PDF}: For formal distribution, printing, archival
\item
  \textbf{HTML}: For web publishing with navigation and search
\item
  \textbf{DOCX}: For collaboration with reviewers who use Word
\item
  \textbf{EPUB}: For reading on mobile devices
\end{itemize}

Write once, publish everywhere. The Makefile handles all conversions.

\subparagraph{Collaborative Writing}\label{collaborative-writing}

Git-friendly plain text enables effective collaboration:

\begin{itemize}
\tightlist
\item
  Multiple authors work on different chapters simultaneously
\item
  Pull requests for review before merging changes
\item
  Clear diff views show exactly what changed
\item
  No binary file conflicts
\end{itemize}

\subsubsection{Getting Started}\label{sec:getting-started}

Quick setup for new users:

\begin{verbatim}
# Clone or create project
cd your-project

# Set up Python environment
uv venv
uv sync --all-extras

# Generate content and build
make all
\end{verbatim}

The document builds to \texttt{whitepaper.pdf} and
\texttt{whitepaper.html}.

For Sphinx output:

\begin{verbatim}
make sphinx-html
\end{verbatim}

Output appears in \texttt{output/html/index.html}.

\subsubsection{Summary}\label{sec:system-summary}

This documentation system addresses the core challenges of technical
writing:

\begin{longtable}[]{@{}
  >{\raggedright\arraybackslash}p{(\linewidth - 2\tabcolsep) * \real{0.3000}}
  >{\raggedright\arraybackslash}p{(\linewidth - 2\tabcolsep) * \real{0.7000}}@{}}
\caption{System Capabilities}\tabularnewline
\toprule\noalign{}
\begin{minipage}[b]{\linewidth}\raggedright
Capability
\end{minipage} & \begin{minipage}[b]{\linewidth}\raggedright
Description
\end{minipage} \\
\midrule\noalign{}
\endfirsthead
\toprule\noalign{}
\begin{minipage}[b]{\linewidth}\raggedright
Capability
\end{minipage} & \begin{minipage}[b]{\linewidth}\raggedright
Description
\end{minipage} \\
\midrule\noalign{}
\endhead
\bottomrule\noalign{}
\endlastfoot
Easy editing & Plain text Markdown/RST with any editor \\
Reproducibility & Python scripts regenerate all derived content \\
Multi-format output & PDF, HTML, DOCX, EPUB from single source \\
LLM compatibility & Structured plain text ideal for AI processing \\
Automation & Single command builds via Makefile \\
Collaboration & Git-friendly with clear diffs \\
Internationalization & Full Korean/CJK support \\
\end{longtable}

The result is a modern documentation workflow that scales from simple
reports to complex technical publications, supports both human and AI
authors, and produces professional output in any required format.

\subsection{Appendix: Project Documentation}\label{sec:appendix}

This appendix contains comprehensive documentation for the Pandoc white
paper authoring system.

\subsubsection{A. Quick Start Guide}\label{sec:quickstart}

\paragraph{Installation (Linux)}\label{installation-linux}

\begin{verbatim}
# System dependencies
sudo apt update
sudo apt install pandoc texlive-full fonts-noto-cjk build-essential

# Python environment
curl -LsSf https://astral.sh/uv/install.sh | sh
cd pandoc_report
uv venv
source .venv/bin/activate
uv pip install matplotlib numpy pandas seaborn scipy pillow tabulate pyyaml
\end{verbatim}

\paragraph{Daily Workflow}\label{daily-workflow}

\begin{verbatim}
# 1. Activate environment
source .venv/bin/activate

# 2. Generate content
python scripts/generate_figures.py
python scripts/generate_tables_v2.py

# 3. Build document
make pdf              # or: make html, make all
\end{verbatim}

\paragraph{Common Commands}\label{common-commands}

{\def\LTcaptype{none} % do not increment counter
\begin{longtable}[]{@{}ll@{}}
\toprule\noalign{}
Command & Description \\
\midrule\noalign{}
\endhead
\bottomrule\noalign{}
\endlastfoot
\texttt{make\ all} & Generate figures, tables, and PDF \\
\texttt{make\ pdf} & Build PDF only \\
\texttt{make\ html} & Build HTML only \\
\texttt{make\ clean} & Remove generated documents \\
\texttt{make\ help} & Show all available commands \\
\end{longtable}
}

\subsubsection{B. Build System Guide}\label{sec:build-guide}

\paragraph{Makefile vs Shell Script}\label{makefile-vs-shell-script}

\textbf{Use ``make pdf`` (Recommended):} - Simpler commands - Automatic
dependency handling - Includes pandoc-crossref filter - Industry
standard

\textbf{Use ``./build\_example.sh`` (For learning):} - Shows exact
Pandoc commands - Useful for debugging - Educational purposes

\paragraph{Build Process}\label{build-process-1}

\begin{verbatim}
1. make figures  →  2. make tables  →  3. make preprocess  →  4. Pandoc build
   (Python)            (Python)           (Replace {{table:..}})    (PDF/HTML)
\end{verbatim}

\subsubsection{C. Korean/CJK Language Support}\label{sec:korean-support}

\paragraph{Setup}\label{setup}

\begin{verbatim}
# Install Korean fonts
sudo apt install fonts-noto-cjk fonts-noto-cjk-extra

# Fonts are pre-configured in metadata.yaml:
# CJKmainfont: "Noto Serif CJK KR"
\end{verbatim}

\paragraph{Usage}\label{usage}

Simply write Korean in your markdown:

\begin{verbatim}
# 서론 {#sec:intro}

한글과 English를 자유롭게 섞어 사용할 수 있습니다.

{{table:tbl:05-korean-data}}
\end{verbatim}

\paragraph{Korean in Matplotlib}\label{korean-in-matplotlib}

\begin{verbatim}
from utils.korean_plot import setup_korean_font
setup_korean_font()

plt.xlabel('한글 레이블')
plt.ylabel('값')
\end{verbatim}

\subsubsection{D. Automatic Table System}\label{sec:automatic-tables}

\paragraph{Table Format (Self-Contained
Objects)}\label{table-format-self-contained-objects}

Each table is a \texttt{.md} file with YAML frontmatter:

\begin{verbatim}
---
id: tbl:03-results
caption: "Experimental results"
description: "Detailed context about the table"
chapter: 3
tags: [experimental, temperature]
---

| Column 1 | Column 2 |
|----------|----------|
| Data     | Data     |
\end{verbatim}

\paragraph{Creating Tables}\label{creating-tables}

\begin{verbatim}
from table_manager import TableManager

tm = TableManager(chapter=3)
tm.save_table(
    data,
    name="my-results",
    caption="Experimental results",
    description="Results from temperature experiments",
    tags=["experimental"]
)
# Saves to: ../_static/tables/chapter-03/table-my-results.md
\end{verbatim}

\paragraph{Using Tables in Chapters}\label{using-tables-in-chapters}

\textbf{Placeholder syntax:}

\begin{verbatim}
## Results

We conducted experiments:

{{table:tbl:03-my-results}}

The results show...
\end{verbatim}

\textbf{Modes:} - \texttt{\{\{table:tbl:03-name\}\}} - Full mode
(includes description) -
\texttt{\{\{table:tbl:03-name\textbar{}inline\}\}} - Inline mode (table
only)

\paragraph{Build Process}\label{build-process-1}

\begin{verbatim}
make pdf  # Automatically replaces placeholders!
\end{verbatim}

The preprocessor: 1. Scans chapters for \texttt{\{\{table:...\}\}}
placeholders 2. Loads tables from \texttt{../\_static/tables/} 3.
Replaces placeholders with actual content 4. Builds document

\paragraph{Manual Preprocessing}\label{manual-preprocessing}

\begin{verbatim}
# Preview changes
python scripts/table_preprocessor.py --dry-run

# Replace placeholders
python scripts/table_preprocessor.py

# List available tables
python scripts/table_preprocessor.py --list-tables
\end{verbatim}

\subsubsection{E. Path Configuration}\label{sec:paths}

\paragraph{Path Rules}\label{path-rules}

All paths in markdown files are \textbf{relative to project root}:

\begin{verbatim}
pandoc_report/              ← Run `make pdf` here
├── chapters/
│   └── 03-results.md       ← Your file
├── ../_static/
│   ├── figures/
│   │   └── plot.png        → ../_static/figures/plot.png
│   └── tables/
│       └── chapter-03/     → {{table:tbl:03-name}}
└── assets/
    └── images/
        └── logo.png         → assets/images/logo.png
\end{verbatim}

\paragraph{Correct Paths}\label{correct-paths}

\begin{verbatim}
<!-- ✅ CORRECT -->
![Plot](../_static/figures/example_plot.png){#fig:plot}
![Logo](assets/images/logo.png)
{{table:tbl:03-results}}

<!-- ❌ WRONG -->
![Plot](../../_static/figures/example_plot.png)
\end{verbatim}

\paragraph{Fixing Path Issues}\label{fixing-path-issues}

\begin{verbatim}
# Fix all image paths
find chapters/ -name "*.md" -exec sed -i 's|../../_static/|../_static/|g' {} \;
\end{verbatim}

\subsubsection{F. Troubleshooting}\label{sec:troubleshooting}

\paragraph{Common Issues}\label{common-issues}

\textbf{Makefile: ``source: not found''} - Fixed! Makefile now uses bash
shell directly - Python called via \texttt{.venv/bin/python}

\textbf{Images not found} - Check paths are from project root:
\texttt{../\_static/figures/plot.png} - Not relative to chapter:
\texttt{../../\_static/figures/plot.png}

\textbf{Korean text shows as boxes}

\begin{verbatim}
sudo apt install fonts-noto-cjk
fc-cache -fv
\end{verbatim}

\textbf{Table placeholder not replaced}

\begin{verbatim}
# List available tables
python scripts/table_preprocessor.py --list-tables

# Check placeholder syntax
{{table:tbl:03-name}}  # ✅ Correct
{{tbl:03-name}}        # ❌ Wrong
\end{verbatim}

\textbf{pandoc-crossref version warning} - Minor warning, usually safe
to ignore - Cross-references still work - Update pandoc-crossref to
match Pandoc version if needed - See:
\url{https://github.com/lierdakil/pandoc-crossref/releases}

\textbf{Python ModuleNotFoundError}

\begin{verbatim}
source .venv/bin/activate
uv pip install matplotlib numpy pandas seaborn scipy pillow tabulate pyyaml
\end{verbatim}

\subsubsection{G. Project Architecture}\label{sec:architecture}

\paragraph{Directory Structure}\label{directory-structure}

\begin{verbatim}
pandoc_report/
├── chapters/              # Markdown chapter files
│   ├── 01-introduction.md
│   ├── 02-methodology.md
│   ├── 03-results.md
│   └── 90-appendix-documentation.md
├── scripts/              # Python scripts
│   ├── generate_figures.py
│   ├── generate_tables_v2.py
│   ├── table_manager.py
│   ├── table_preprocessor.py
│   └── utils/
│       └── korean_plot.py
├── ../_static/               # Generated content
│   ├── figures/
│   └── tables/
│       └── chapter-{NN}/
├── assets/               # Static resources
│   └── images/
├── templates/            # Pandoc templates
├── references/           # Bibliography
│   └── references.bib
├── metadata.yaml         # Document metadata
├── Makefile             # Build automation
├── pyproject.toml       # Python dependencies
└── README.md            # Quick start guide
\end{verbatim}

\paragraph{File Organization}\label{file-organization}

\textbf{Source files (commit to git):} - \texttt{chapters/*.md} - Your
content - \texttt{scripts/*.py} - Analysis and generation -
\texttt{metadata.yaml} - Document configuration - \texttt{Makefile} -
Build system - \texttt{pyproject.toml} - Dependencies

\textbf{Generated files (don't commit):} -
\texttt{../\_static/figures/*.png} - Generated plots -
\texttt{../\_static/tables/**/*.md} - Generated tables - \texttt{*.pdf},
\texttt{*.html} - Build outputs - \texttt{.venv/} - Virtual environment

\subsubsection{H. Advanced Features}\label{sec:advanced}

\paragraph{Cross-References}\label{cross-references-1}

\begin{verbatim}
# Chapter {#sec:chapter}

## Results {#sec:results}

![Plot](../_static/figures/plot.png){#fig:plot width=80%}

| A | B |
|---|---|
| 1 | 2 |

: Caption {#tbl:data}

$$E = mc^2$$ {#eq:einstein}

See Section @sec:results, Figure @fig:plot,
Table @tbl:data, and Equation @eq:einstein.
\end{verbatim}

\paragraph{Citations}\label{citations}

\begin{verbatim}
Recent work [@smith2020; @jones2021] shows...
@smith2020 demonstrated that...
\end{verbatim}

Add entries to \texttt{references/references.bib}:

\begin{verbatim}
@article{smith2020,
  title={Example Article},
  author={Smith, John},
  journal={Example Journal},
  year={2020}
}
\end{verbatim}

\paragraph{Custom Pandoc Options}\label{custom-pandoc-options}

Edit \texttt{Makefile} to add options:

\begin{verbatim}
COMMON_OPTS = --number-sections --toc --toc-depth=3 \
              --highlight-style=tango \
              --variable=geometry:margin=1in
\end{verbatim}

\paragraph{Multiple Output Formats}\label{multiple-output-formats}

\begin{verbatim}
make pdf          # PDF via LaTeX
make html         # Standalone HTML
make docx         # Microsoft Word
make all-formats  # All formats
\end{verbatim}

\subsubsection{I. Best Practices}\label{sec:best-practices}

\paragraph{1. Version Control}\label{version-control}

\begin{verbatim}
# Commit source files
chapters/
scripts/
metadata.yaml
Makefile
pyproject.toml

# Don't commit generated files
../_static/
*.pdf
*.html
.venv/
\end{verbatim}

\paragraph{2. Table Management}\label{table-management}

\begin{itemize}
\tightlist
\item
  Use descriptive names: \texttt{table-temperature-results.md}
\item
  Include descriptions for context
\item
  Tag tables for organization
\item
  Regenerate when data changes
\end{itemize}

\paragraph{3. Figure Quality}\label{figure-quality}

\begin{verbatim}
# Save at high DPI for publication
plt.savefig('../_static/figures/plot.png', dpi=300, bbox_inches='tight')
\end{verbatim}

\paragraph{4. Modular Chapters}\label{modular-chapters}

\begin{itemize}
\tightlist
\item
  One chapter per file
\item
  Use meaningful section IDs
\item
  Keep chapters focused
\end{itemize}

\paragraph{5. Reproducibility}\label{reproducibility}

\begin{verbatim}
# Document your workflow
make clean-all    # Remove everything
make all          # Rebuild from scratch
\end{verbatim}

\subsubsection{J. Quick Reference}\label{sec:reference}

\paragraph{Build Commands}\label{build-commands}

\begin{verbatim}
make pdf          # Build PDF
make html         # Build HTML
make all          # Everything
make clean        # Remove outputs
make help         # Show options
\end{verbatim}

\paragraph{Python Scripts}\label{python-scripts}

\begin{verbatim}
# Generate content
python scripts/generate_figures.py
python scripts/generate_tables_v2.py
python scripts/my_analysis.py

# Manage tables
python scripts/table_preprocessor.py --list-tables
python scripts/table_preprocessor.py --dry-run
\end{verbatim}

\paragraph{Markdown Syntax}\label{markdown-syntax}

\begin{verbatim}
# Heading {#sec:id}
![Caption](path){#fig:id width=80%}
: Caption {#tbl:id}
$$equation$$ {#eq:id}
{{table:tbl:03-name}}
[@citation]
@fig:id, @tbl:id, @sec:id
\end{verbatim}

\paragraph{File Paths}\label{file-paths}

{\def\LTcaptype{none} % do not increment counter
\begin{longtable}[]{@{}ll@{}}
\toprule\noalign{}
Resource & Path Format \\
\midrule\noalign{}
\endhead
\bottomrule\noalign{}
\endlastfoot
Figures & \texttt{../\_static/figures/name.png} \\
Tables & \texttt{\{\{table:tbl:NN-name\}\}} \\
Static images & \texttt{assets/images/name.png} \\
Bibliography & \texttt{references/references.bib} \\
\end{longtable}
}

\subsubsection{K. Resources}\label{sec:resources}

\paragraph{Documentation}\label{documentation}

\begin{itemize}
\tightlist
\item
  Pandoc Manual: \url{https://pandoc.org/MANUAL.html}
\item
  Pandoc-Crossref: \url{https://lierdakil.github.io/pandoc-crossref/}
\item
  Markdown Guide: \url{https://www.markdownguide.org/}
\item
  UV: \url{https://github.com/astral-sh/uv}
\item
  Matplotlib: \url{https://matplotlib.org/}
\end{itemize}

\paragraph{Project Files}\label{project-files}

\begin{itemize}
\tightlist
\item
  \texttt{scripts/table\_manager.py} - TableManager API
\item
  \texttt{scripts/table\_preprocessor.py} - Preprocessor
\item
  \texttt{scripts/demo\_table\_workflow.py} - Working demo
\item
  \texttt{Makefile} - Build system reference
\end{itemize}

\paragraph{Getting Help}\label{getting-help}

\begin{verbatim}
# Show available make targets
make help

# List available tables
python scripts/table_preprocessor.py --list-tables

# Check versions
pandoc --version
python --version
make --version
\end{verbatim}

\subsubsection{L. Summary of Features}\label{sec:feature-summary}

\paragraph{✅ Implemented Features}\label{implemented-features}

\begin{enumerate}
\def\labelenumi{\arabic{enumi}.}
\tightlist
\item
  \textbf{Automated Build System}

  \begin{itemize}
  \tightlist
  \item
    Makefile with dependency management
  \item
    Automatic figure/table generation
  \item
    Multi-format output (PDF, HTML, DOCX)
  \end{itemize}
\item
  \textbf{Korean/CJK Language Support}

  \begin{itemize}
  \tightlist
  \item
    Full UTF-8 support
  \item
    Korean fonts configured
  \item
    Mixed language documents
  \item
    Korean matplotlib plots
  \end{itemize}
\item
  \textbf{Automatic Table System}

  \begin{itemize}
  \tightlist
  \item
    Self-contained table objects
  \item
    YAML frontmatter metadata
  \item
    Placeholder replacement
  \item
    Multiple modes (full/inline)
  \end{itemize}
\item
  \textbf{Reproducible Workflow}

  \begin{itemize}
  \tightlist
  \item
    Python-based generation
  \item
    Version control friendly
  \item
    UV virtual environment
  \item
    Documented dependencies
  \end{itemize}
\item
  \textbf{Professional Output}

  \begin{itemize}
  \tightlist
  \item
    Publication-quality figures (300 DPI)
  \item
    Automatic numbering
  \item
    Cross-references
  \item
    Bibliography management
  \end{itemize}
\item
  \textbf{Developer-Friendly}

  \begin{itemize}
  \tightlist
  \item
    Modular architecture
  \item
    Clear documentation
  \item
    Example scripts
  \item
    Troubleshooting guides
  \end{itemize}
\end{enumerate}

This system provides a complete, professional authoring environment for
technical white papers with automated content generation and
multi-language support.


\end{document}
